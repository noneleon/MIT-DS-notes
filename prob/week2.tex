\section*{week 2 conditioning and independence}

\subsection*{conditioning and bayes' rule}

\textcolor{blue}{conditional prob:}
$P(A|B)=$prob of A, given that B occured

$P(A|B)=\frac{P(AB)}{P(B)}$ defined only when $P(B)\ge 0$

\textcolor{blue}{the multiplication rule}

$P(AB)=P(A)P(B|A)$

$P(\cap_i A_i)=P(A_1)\prod_{i=2}^n P(A_i|\cap_{i-1} A_{i})$


\textcolor{red}{total prob theorem}
$P(B)=\sum_i P(A_i)P(B|A_i)$

\textcolor{red}{bayes' rule}
$P(A_i|B)=\frac{P(A_i)P(B|A_i)}{\sum_j P(A_j)P(B|A_j)}$


\subsection*{independent}

\textcolor{blue}{independence of two events}
$P(AB)=P(A)P(B)$

\textcolor{blue}{conditional independence}

conditonal independence, given C, is defined as independence under the prob law $P(.|C)$

$P(AB|C)=P(A|C)P(B|C)$

\textcolor{red}{reliability}

\begin{itemize}
    \item chuan $p(chuan)=\prod_i p_i$
    \item bing  $p(bing)=1-\prod_i(1-p_i)$
\end{itemize}